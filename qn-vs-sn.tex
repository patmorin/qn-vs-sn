\documentclass{patmorin}
\listfiles
\usepackage{pat}
\usepackage{paralist}
\usepackage{dsfont}  % for \mathds{A}
\usepackage[utf8x]{inputenc}
\usepackage{skull}
\usepackage{paralist}
\usepackage{graphicx}
\usepackage[noend]{algorithmic}

\usepackage[normalem]{ulem}
\usepackage{cancel}
\usepackage{enumitem}

\usepackage{todonotes}

\usepackage[longnamesfirst,numbers,sort&compress]{natbib}

\usepackage[mathlines]{lineno}
\setlength{\linenumbersep}{2em}
% \linenumbers
% \rightlinenumbers
% \linenumbers
\newcommand*\patchAmsMathEnvironmentForLineno[1]{%
 \expandafter\let\csname old#1\expandafter\endcsname\csname #1\endcsname
 \expandafter\let\csname oldend#1\expandafter\endcsname\csname end#1\endcsname
 \renewenvironment{#1}%
    {\linenomath\csname old#1\endcsname}%
    {\csname oldend#1\endcsname\endlinenomath}}%
\newcommand*\patchBothAmsMathEnvironmentsForLineno[1]{%
 \patchAmsMathEnvironmentForLineno{#1}%
 \patchAmsMathEnvironmentForLineno{#1*}}%
\AtBeginDocument{%
\patchBothAmsMathEnvironmentsForLineno{equation}%
\patchBothAmsMathEnvironmentsForLineno{align}%
\patchBothAmsMathEnvironmentsForLineno{flalign}%
\patchBothAmsMathEnvironmentsForLineno{alignat}%
\patchBothAmsMathEnvironmentsForLineno{gather}%
\patchBothAmsMathEnvironmentsForLineno{multline}%
}


\newcommand{\coloured}[2]{{\color{#1}{#2}}}
\newenvironment{colourblock}[1]{\color{#1}}{}

\newcommand{\condref}[1]{(C\ref{#1})}

% Taken from
% https://tex.stackexchange.com/questions/42726/align-but-show-one-equation-number-at-the-end
\newcommand\numberthis{\addtocounter{equation}{1}\tag{\theequation}}


\setlength{\parskip}{1ex}


\DeclareMathOperator{\diam}{diam}
\DeclareMathOperator{\tw}{tw}
\DeclareMathOperator{\gm}{gm}
\DeclareMathOperator{\gs}{gs}
\DeclareMathOperator{\qn}{qn}
\DeclareMathOperator{\sn}{sn}
\DeclareMathOperator{\stw}{stw}
\DeclareMathOperator{\ltw}{ltw}
\DeclareMathOperator{\pw}{pw}
\DeclareMathOperator{\lpw}{lpw}
\DeclareMathOperator{\lhptw}{lhp-tw}
\DeclareMathOperator{\lhppw}{lhp-pw}

\DeclareMathOperator{\x}{x}
\DeclareMathOperator{\depth}{d}
\DeclareMathOperator{\sh}{cbt}
\DeclareMathOperator{\cbt}{cbt}
\DeclareMathOperator{\sgn}{sgn}
\DeclareMathOperator{\dc}{dc}
\DeclareMathOperator{\afci}{\overline{\chi}_\pi}
\DeclareMathOperator{\afcn}{\dot{\chi}_\pi}

\newcommand{\ellt}{{\lfloor\ell/2\rfloor}}

\title{\MakeUppercase{Queue Number versus Stack Number}\thanks{This research was partly funded by NSERC.}}
\author{Pat Morin%
    \thanks{School of Computer Science, Carleton University}\quad
    and Friends}

\date{}

\DeclareMathOperator{\ddiv}{div}
\DeclareMathOperator{\hist}{h}

\newcommand{\colored}[2]{{\color{#1}#2}}

\usepackage{tabularx}

\DeclareMathOperator{\ci}{\overline{\pi}}

\begin{document}

% \begin{titlepage}
\maketitle

\begin{abstract}
    These are some notes I'm keeping on trying to show that queue number is (or isn't) bounded by stack number.
\end{abstract}
% \end{titlepage}

% \pagenumbering{roman}
% \tableofcontents
%
% \newpage
% \pagenumbering{arabic}



\section{Introduction}

A lemma of \citet{dujmovic.wood:stacks} states that, if queue-number is bounded by stack-number, then queue-number is bounded by a polynomial function of stack number. More precisely, if there exists a function $f:\N\to\N$ such that $\qn(G)\le f(\sn(G))$ for all graphs $G$ then there exists a constant $c$ such that $\qn(G)\in O((\sn(G))^c)$ for all graphs $G$.

\begin{lem}\label{rainbow_alternative}
    Let $M$ be a perfect matching with edge set $E(M):=\{s_it_i:i\in\{1,\ldots,n\}\}$ and let $<$ be a total order on $V(M)$.  Then
    \begin{compactenum}[(i)]
        \item $M$ contains a rainbow of size at least $q$ or
        \item there exists $I\subset\{1,\ldots,n\}$, $|I|\ge k/(2q)$ such that $s_i < s_j$ if and only if $t_i< t_j$ for each $i,j\in I$.
    \end{compactenum}
\end{lem}

\begin{proof}
    Partition the edges of $M$ into two classes, those for which $a_i<b_i$ and those for which $a_i>b_i$.  Assume, without loss of generality that the first class has size at least $|M|/2$.  Next apply Dilworth's Theorem to the partial order $(\prec,\{1,\ldots,k\})$ in which $i\prec j$ if and only if $a_i<a_j$ and $b_i < b_j$.  Any antichain in $\prec$ is a rainbow and any chain in $\prec$ is the set $I$ in the second alternative.
\end{proof}

\begin{lem}\label{rainbow_alternative_paths}
    Let $G$ be a graph consisting of $n$ vertex-disjoint paths $P_1,\ldots,P_n$, each of length $\ell$ and with $P_i$ beginning at $s_i$ and ending at $t_i$, for each $i\in\{1,\ldots,k\}$.  Then, for any $q$-queue layout $(<,\varphi)$ of $G$, there exists $J\subseteq\{1,\ldots,n\}$, $|J|\ge n/(2q)^\ell$ such that $s_i<s_j$ if and only if $t_i<t_j$ for each $i,j\in J$.
\end{lem}

\begin{proof}
    The proof is by induction on $\ell$.  If $\ell=0$, then $s_i=t_i$ for each $i\in\{1,\ldots,n\}$ and the result is trivially true, by taking $J:=\{1,\ldots,n\}$.

    If $\ell\ge 1$, then apply \cref{rainbow_alternative} to the matching $M$ with edge set $E(M):=\{s_is_i':i\in\{1,\ldots,k\}\}$, where $s_i'$ denotes the second vertex in $P_i$.  The fact that $(<,\varphi)$ is a $q$-queue layout rules out alternative \cref{rainbow_alternative}(i).  Therefore, let $I$, $|I|\ge k/(2q)$, be the set determined by \cref{rainbow_alternative}(ii).  For each $i\in I$, let $P'_i$ be the path obtained by removing $s_i$ from $P_i$ (so that $P_i'$ begins at $s_i'$ and ends at $t_i$).  Now apply the inductive hypothesis to the set of paths $\{P_i':i\in I\}$.  This gives a set $J\subseteq I$ of size $|J|\ge |I|/(2q)^{\ell-1}\ge n/(2q)^\ell$ for which $s_i' < s_j'$ if and only if $t_i < t_j$ for each $i,j\in J$.  By \cref{rainbow_alternative}(ii), $a_i < a_j$ if and only if $a_i' < a_j'$, for each $i,j\in I$.  By transitivity, for each $i\in K$, $s_i < s_j$ if and only if $t_i< t_j$, as required.
\end{proof}

Here is a blocked variant of the Erd\H{o}s-Szekeres Theorem that may or may not be true:
\begin{lem}\label{block_erdos_szekeres}
    Let $n:=cr^2$ for some integers $r$ and $c$ and let $\pi_1,\ldots,\pi_n$ be a permutation of $\{1,\ldots,n\}$.  Then there exists integers $1\le k_1<k_2<\cdots<k_r\le cr$ and  $i_1<i_2<\cdots< i_r$ such that
    \begin{compactenum}[(i)]
        \item $i_x\in \{(k_x-1)r+1,\ldots,k_xr\}$ for each $x\in\{1,\ldots,r\}$ and
        \item $\pi_{i_1}<\pi_{i_2}<\cdots<\pi_{i_r}$ or  $\pi_{i_1}>\pi_{i_2}>\cdots>\pi_{i_r}$
    \end{compactenum}
\end{lem}


Let $Q_d$ be the \emph{$d$-cube}, the graph with vertex set $V(Q_d):=\{0,\ldots,2^d-1\}$ and that contains the edge $xy$ if and only if the binary representation of $x$ and $y$ differ in exactly one bit.  For two graphs $G$ and $H$, the \emph{direct product} $G\times H$ is the graph whose vertex set is the cartesian product $V(G\times H):=V(G)\times V(H)$ and whose edge set and edge with endpoints $(x_1,y_1)$ and $(x_2,y_2)$ if and only if $x_1x_2\in E(G)$ and $y_1y_2\in E(H)$.

\begin{lem}\label{routing}
    For any $\alpha,\beta \ge 0$, there exists $r:=r(\alpha,\beta)$ and $\gamma:=\gamma(\alpha,\beta)$ such that the following is true:  Let $P:=p_0,\ldots,p_r$ be a path and let $R:=R(d,\alpha,\beta)$ be the supergraph of $Q_d\times P_r$ that also contains the edges of the path $(x,p_0),\ldots,(x,p_r)$ for each $x\in V(Q_d)$.  Let $S$ be a subset of $V(Q_d)$ with $|S|\ge 2^d/(\alpha d^{\beta})$.  Let $T=\{n-1-s: s\in S\}$.  Then $R$ contains $s\ge \lfloor \alpha 2^{\gamma d}\rfloor$ vertex-disjoint length-$r$ paths $P_1,\ldots,P_s$ such that,
    \begin{compactenum}
        \item for each $i\in\{1,\ldots,s\}$, $P_i$ begins at $(s_i,p_0)$ for some $s_i\in S$ and ends at $(t_1,p_r)$ some $t_i\in T$ and
        \item for each $i,j\in\{1,\ldots,s\}$, $s_i < s_j$ if and only if $t_i < t_j$.
    \end{compactenum}
\end{lem}

\begin{proof}
    Induction on the dimension, with two cases.  Either the dimension is balanced, so we get a recurrence like $T(d,m)\ge 2T(d-1,m/3)$ or the dimension is unbalanced, so we get a recurrence like $T(n)=T(d-1,2n/3)$. The first one naturally resolves to a polynomial. The second one gives us a denser problem, which can only happen $O(\beta\log d+\log\alpha)=O(\beta)$ times.
\end{proof}



Let $c$ be the constant in \cref{block_erdos_szekeres}, rounded up to the next integer power of $2$ (so $c:=2^{\lceil\log_2 c'\rceil}$ where $c'$ is denoted by $c$ in the statement of \cref{block_erdos_szekeres}).  Let $r:=2^d$ for some positive integer $d$ and let $n:=cr$ where $c$ is the constant given by \cref{block_erdos_szekeres}.  Now we construct the following graph $G$ from the following pieces:
\begin{enumerate}
    \item $G$ contains the set of $n$ length-$2$ paths described by  \cref{rainbow_alternative_paths} for $\ell:=2$;
    \item $G$ contains the graph $R(d+\log_2 c,c)$ described in \cref{routing};
    \item For each $k\in\{0,\ldots,cr-1\}$ and each $i\in\{1,\ldots,r\}$, $G$ contains the edge with endpoints $(k,p_0)$ and $s_{kr+i}$; and
    \item For each $k\in\{0,\ldots,cr-1\}$ and each $i\in\{1,\ldots,r\}$, $G$ contains the edge with endpoints $(k,p_r)$ and $t_{kr+i}$; and
\end{enumerate}

\bibliographystyle{plainurlnat}
\bibliography{qn-vs-sn}
\end{document}

\section{Old Stuff}

An old result of \citet[Theorem 8]{dujmovic.wood:stacks} shows that queue number is (polynomially) bounded by by stack number if and only if bipartite 3-stack graphs have bounded queue number.  Even forgetting the bipartite condition, this means we can restrict ourselves to graphs made up of three outerplanar graphs $G_1$, $G_2$, and $G_3$ having a common outer face $v_1,\ldots,v_n$.

If we're trying to show that queue number is not bounded by stack number then we will need all three of these triangulations.  Why? Because two triangulations of $v_1,\ldots,v_n$ form a (Hamiltonian) planar graph.  We already know that any planar graph has queue number at most $49$.

I conjecture that the expected queue number of the random (multi)graph $G$ obtained by taking each of $G_1,\ldots,G_3$ to be a random triangulation of $v_1,\ldots,v_n$ is unbounded.  Here, a \emph{random triangulation} of a $n$-vertex cycle $C$ is the outerplanar graph with outer face $C$ and whose weak dual is a uniformly random binary search tree with $n-2$ nodes.  [There is a bijection between these triangulations and full binary trees with $n-1$ leaves, so the number of such triangulations is the $(n-2)$-th Catalan number $C_{n-2}$.]

Even if this turns out to be false or if the result can be proven with a deterministic construction, I think this model of random graph is interesting and worth studying.  In general, we could define the ``random stack-number-$k$ graph'' $R_{n,k}$ obtained by taking the union of $k$ random triangulations of the cycle $v_1,\ldots,v_n$ and ask questions about it.
\begin{compactenum}
    \item What is the expected maximum degree of $R_{n,k}$?
    \item What is the expected diameter of $R_{n,k}$?
    \item What is the expected treewidth of $R_{n,k}$?
    \item What is the expected queue number of $R_{n,k}$?
\end{compactenum}

Some conjectures about the answers to these questions:

\begin{enumerate}
    \item For $k=1$, the answers are $\Theta(\log n)$, $\Theta(\sqrt{n})$, $2$, and $O(1)$.  (The last two are facts, not conjectures)

    \item For $k=2$, the answers are $\Theta(\log n)$, $\Theta(\sqrt{n})$, $\Theta(\sqrt{n})$, and $O(1)$, respectively.  (Only the last one is a fact, the rest are conjectures.)

    \item For $k=3$ the answers are $\Theta(\log n)$, $\Theta(n^{1/3})$, $\Theta(n^{2/3})$ and $\Omega(1)$. (All conjectures.)
\end{enumerate}

The question about maximum-degree is really just about uniformly random binary trees.  It asks what is the longest path in such a tree that repeatedly goes from a node to its right child.  It feels like we can use symmetry to treat this the same as a random walk in an arbitrary binary tree.  In an arbitrary binary tree, the probability that a random walk takes more than $\log n+k$ steps is at most $2^{-k}$.  I guess this could easily be made precise and settles the first question for any constant $k$.

Here's a very rough intuition about the diameter (for $k=1$).  Since the height of the random tree $T$ that defines the triangulation $R_1$ is $\Theta(\sqrt{n})$, most nodes of $T$ have a child that is the root of a subtree of size $O(\sqrt{n})$.  So most of the time, the triangle $v_1v_iv_n$ of $R_1$ has $\max\{i, n-i\}\in O(\sqrt{n})$, say $i\in O(\sqrt{n})$.  Then the node furtest from $v_1$ is at distance least one greater than the node furthest from $\{v_i,v_n\}$ on the subgraph induced by $v_i,\ldots,v_n$.  This is a subgraph of size $n-O(\sqrt{n})$.  I guess this argument is basically showing that there exists a cutset of size $2$ that separates a vertex from a part of the graph of size $n-\sqrt{n}$.

Here's another approach for the diameter in the case $k=1$.  With high probability, the tree $T$ contains a root-to-leaf path $P$ of length $\Omega(\sqrt{n})$.  With high probablity, $P$ contains $\Omega(\sqrt{n})$ \emph{alternations}; subpaths of the form $xyz$ where $y$ is the right child of $x$ and $z$ is the left child of $y$ (or \emph{vice-versa} with the roles of left and right reversed).  This means there exists a vertex $v$ of $R_1$ such that $\{v_1,v_n\}$ is separated from $v$ by a sequence $\{a_1,b_1\},\ldots,\{a_k,b_k\}$ of disjoint $2$-vertex cutsets, where $k\in\Omega(\sqrt{n})$.  So every path from $v_1$ to $v$ must contain at least one of $a_i$ or $b_i$ for each $i\in\{1,\ldots,k\}$.  Therefore, $\diam(G)\ge k+1\in \Omega(\sqrt{n})$.

Both of the preceding arguments seems to break down immediately, even for $k=2$.  Note that understanding diameter is a prerequisite for understanding treewidth since (at least for $k=2$) $\tw(R_2) \in O(\diam(R_2))$ since $R_2$ is planar.
% In fact, upon reflection, I think that $\diam(R_2)\in O(\log n)$.


\subsection{Another Candidate}

Another candidate graph is the following: $G_1\cup G_2$ form an $n\times n$ grid.  (This has a realization as a 2-page graph where $v_1,\ldots,v_n$ is the snake-like path through the rows of the grid, $G_1$ contains the vertical grid edges from row $i$ to row $i+1$ for odd $i$, and $G_2$ contains the remaining vertical grid edges.)  The graph $G_3$ is the random graph $R_1$ described above.

First we'd like to see if the treewidth of this graph is large.  We might try to do this as follows:
\begin{compactenum}
    \item Fix a separation $(A,B)$ of $G_1\cup G_2$ of order $O(\sqrt{n})$.
    \item This separation divides $v_1,\ldots,v_n$ into $O(sqrt(n))$ intervals.  Some of these are in $A\setminus B$ and some of these are in $B\setminus A$.
    \item The
\end{compactenum}


\subsection{Another Candidate: The $3$-Ary Hypertwist}

Let $N:=3^{d-1}$ for some integer $d\ge 1$.  Given a sequence $V:=x_1,\ldots,x_{N},y_1,\ldots,y_N,z_1,\ldots,z_N$ of $3^d$ distinct vertices we define the \emph{$3$-ary hypertwist} $H_V:=H$ over $V$ as follows (see \cref{hypertwist}):
\begin{compactenum}
    \item $V(H_V):=\{V\}$;
    \item $E(H_V)$ contains $\bigcup_{i=1}^N\{x_iy_i, y_iz_i, x_iz_{n-i+1}\}$; and
    \item If $d \ge 2$ then $H_d$ contains $H_{x_1,\ldots,x_N}$, $H_{y_1,\ldots,y_N}$ and $H_{z_1,\ldots,z_N}$.
\end{compactenum}

\begin{figure}
    \begin{center}
        \includegraphics{figs/hypertwist-3}
    \end{center}
    \caption{The graph $H_3$. In this drawing the vertices appear in the order $x_1,\ldots,X_N,y_N,\ldots,y_1,z_1,\ldots,z_N$.}
    \label{hypertwist}
\end{figure}

We call $H_V$ a \emph{hypertwist of dimension $d$} and we denote it by $H_d$. For convenience, $H_0$ is the graph consisting of a single vertex. The following is easily proven by induction on $d$ (and \cref{hypertwist} illustrates the ordering used in the stack layout):

\begin{lem}\label{hypertwist_sn}
    $\sn(H_d) \le 3d-2$.
\end{lem}

If we can prove the following conjecture then we are done:

\begin{conj}\label{hypertwist_qn_conjecture}
    There exists an integer $d_0$ and real $\alpha >0$ such that $\qn(H_d)\ge 3^{\alpha d}$ for all $d\ge d_0$.
\end{conj}

Explanation:  A lemma of \citet{dujmovic.wood:stacks} states that, if queue-number is bounded by stack-number, then queue-number is bounded by a polynomial function of stack number. More precisely, if there exists a function $f:\N\to\N$ such that $\qn(G)\le f(\sn(G))$ for all graphs $G$ then there exists a constant $c$ such that $\qn(G)\in O((\sn(G))^c)$ for all graphs $G$.  By \cref{hypertwist_sn}, $\sn(H_d)=O(d)$, but \cref{hypertwist_qn_conjecture} asserts that $\qn(G)\ge 3^{\Omega(d)}$.  We are done because $\qn(H_d)\ge 3^{\Omega(d)} \not\in O(d^c)=\sn(H_d)$ for any constant $c$.


This graph behaves a lot like a $3$-ary hypercube.

\begin{compactenum}
    \item Each vertex of $H_d$ has degree $2d$.
    \item $H_d$ has diameter $d$ (use the coordinate-based routing algorithm, like in a hypercube).
    \item I think (but would have to prove) that any $m$-vertex subgraph of $H_d$ has average degree at most $2\log_3 m$.  (Heath, Leighton, and Rosenberg proved this for the $3$-ary hypercube.)
\end{compactenum}


Let's start with a basic and well-known lemma:

\begin{lem}\label{rainbow_alternative}
    Let $M$ be a perfect matching with edge set $E(M):=\{a_ib_i:i\in\{1,\ldots,k\}\}$ and let $<$ be a total order on $V(M)$.  Then $M$ contains a rainbow of size at least $q$ or there exists $I\subset\{1,\ldots,k\}$, $|I|\ge k/q$ such that $a_i < a_j$ if and only if $b_i< b_j$ for each $i,j\in I$.
\end{lem}

\begin{proof}
    Apply Dilworth's Theorem to the partial order $(\prec,\{1,\ldots,k\})$ in which $i\prec j$ if and only if $a_i<a_j$ and $b_i < b_j$.  Any antichain in $\prec$ is a rainbow and any chain in $\prec$ is the set $I$ in the second alternative.
\end{proof}


\begin{lem}\label{x-to-z}
    Let $(<,\varphi)$ be a $<q$-queue layout of $H_d$. Then, for any $I\subset\{1,\ldots,N\}$, there exists $I'\subseteq I$, $|I'|\ge|I|/q^2$ such that, for each $i,j\in I$, $x_i<y_j$ if and only if $y_i < y_j$ and $z_i < z_j$.
\end{lem}

\begin{proof}
    Since $(<,\varphi)$ is a $<q$-queue layout, $H_d$ does not contain any rainbow of size $q$ with respect to $<$.  Apply \cref{rainbow_alternative} twice, once using the matching $\{x_iy_i: i\in I\}$ to obtain a set $I_1$ of size at least $|I|/q$ and a second time to the matching $\{y_i z_i: i\in I_1\}$ to obtain the set $I'$ of size at least $|I|/q^2$.
\end{proof}

\begin{lem}
    Let $(<,\varphi)$ be a $<q$-queue layout of $H_d$. Then there exists $I\subset\{1,\ldots,N\}$ with the following properties:
    \begin{compactenum}
        \item $|I'|\ge N/q^3$;
        \item for each $i\in I$, $\overline\imath\in I$;
        \item for each distinct $i,j\in \{x\in I:x < N/2\}$, $x_i < x_j$ if and only if $x_{\overline\imath}< x_{\overline\jmath}$; and
        \item $i,j\in I$, $x_i<y_j$ if and only if $y_i < y_j$ and $z_i < z_j$.
    \end{compactenum}
\end{lem}

\begin{proof}
    First apply \cref{rainbow_alternative} to the matching $\{x_ix_{\overline\imath}: i\in\{1,\ldots,N\}\}$ then apply \cref{x-to-z} to the resulting set.
\end{proof}

% The lemmma stated in this comment is false.
% For $I\subseteq\{1,\ldots,N\}$, let $x_I:=\{x_i:i\in I\}$, define $y_I$ and $z_I$ similarly, and define $G_I:=G[x_I\cup y_I\cup z_I]$.  We say that such a set $I$ is \emph{symmetric} if $\overline{v}\in V(G_I)$ for each $v\in V(G_I)$.
%
% \begin{lem}
%     If there exists a $k$-element symmetric set $I\subset\{1,\ldots,N\}$ such that $x_I\cup y_I$ is separated from $z_I$ with respect to $<$, then $H_d$ contains a rainbow of size $k^{c}$.
% \end{lem}
%
% \begin{proof}
%     Here are the steps in the proof:
%     \begin{compactenum}
%         \item Use pigeonhole to find $I_1\subseteq I$ of size at lest $k/2$ such that $x_i< y_i$ for each $i\in I_1$.
%
%         \item Use Dilworth's Theorem and the existence of each edge $x_ix_{\overline\imath}$ to find $I_2\subseteq I_1$ such that $x_{\{i\in I_2:i\le N/2\}}$ and $x_{\{i\in I_2:i\ge N/2\}}$ have the same order (where we match $x_i$ with $x_{\overline{\imath}}$).
%
%         \item Use Dilworth and the existence of the edge $x_iy_i$ to find $I_3\subseteq I_2$ such that $x_{I_2}$ and $y_{I_2}$ have the same ordering (where we match $x_i$ with $y_i$).
%
%         % \item Use Dilworth and the existence of the edge $z_iz_{\overline\imath}$ to find $I_4\subseteq I_3$ such that $z_{\{i\in I_3:i\le N/2\}}$ and $z_{\{i\in I_3:i\ge N/2\}}$ have the same order (where we match $z_i$ with $z_{\overline{\imath}}$).
%     \end{compactenum}
%     Now we're done.  Let $I_4':=\{i\in I_4:i\le N/2\}$ and let $\overline{I}_4':= I_4\setminus I_4'$.
%     If $y_{I_4'}$ and $z_{I_4'}$ don't contain a big rainbow then $y_{I_4}$ and $z_{I_4'}$ contain a big twist.  By Step~2 above, $x_{I_4'}$ and $y_{I_4'}$ have the same order.  By Step~1 above, $x_{I_4'}$ and $y_{\overline{I}_4'}$ have the same order.  This implies that the edge set $\{x_{\overline{\imath}}z_i:i\in I_4'\}$ form a big rainbow.
% \end{proof}

\begin{lem}
  For any $q$-queue layout $(<,\varphi)$  of $H_d$, there exists $I\subseteq\{1,\ldots,N/2\}$ and an integer $s\in\{1,\ldots,6\}$ such that:
  \begin{compactenum}[(i)]
    \item $|I|\ge N/q^{O(1)}-f(q)$, for some fixed $f:\N\to\N$;

    \item for each $i,j,\in\{1,\ldots,N/2\}$, the permutation of $R_i:=x_i,x_{\overline{\imath}},y_i,y_{\overline{\imath}},z_i,z_{\overline{\imath}}$ defined by $<$ is the same as the permutation of $R_j:=x_j,x_{\overline{\jmath}},y_j,y_{\overline{\jmath}},z_j,z_{\overline{\jmath}}$ defined by $<$; and

    \item for each distinct $i,j\in\{1,\ldots,N/2\}$, $R_i$ is an $s'$-shift of $R_j$ for some $s'\in\{-s,s\}$.
  \end{compactenum}
\end{lem}

\begin{proof}[Proof Sketch]
  First classify $i\in\{1,\ldots,N/2\}$ based on the permutation of $R_i$ (there are only $6!$) and then on the edge-colouring of (a spanning tree of) $H_d[R_i]$ (there are only $q^5$).  Taking the biggest class gives us a set $I_0$ of size at least $N/(2\cdot 6!\cdot q^5)$ that satifies (ii).  Now, classify each pair $i,j\in I_0$ based on how $R_i$ and $R_j$ interleave.  There are no more than $2^{12}$ categories and $6$ of them are $s$-shifts.  Suppose that one of the categories, $C$, that is not an $s$-shift has size greater than $f(q)$ for some (very fast growing function $f$).  Then, by Ramsey's Theorem there exists $J\subseteq I_0$ of size at least $q^c$ such that each pair in $J$ is in category $C$. Now argue that results in a rainbow of size greater than $q$.

  Conclude that these ``non-shift'' categories have total size at most $2^{12}f(q)$, so one of the $6$ shift categories has size at least $(|I_0|-2^{12}f(q))/6$.
\end{proof}



\begin{lem}
    If there exists a $k$-element symmetric set $I\subset\{1,\ldots,N\}$ such that $x_I\cup z_I$ is separated from $y_I$ with respect to $<$, then $H_d$ contains a rainbow of size $k^{1/4}/sqrt{2}$.
\end{lem}

\begin{proof}
    Similar to the previous proof, except that
\end{proof}


\begin{obs}\label{cant_separate2}
    If there exists a set $I\subseteq\{x_{N/2+1},\ldots,x_{N},y_{1},y_{N/2}\}$ such that $I$ and $\overline{I}$ are separated with respected to $<$, then $G$ contains a rainbow of size at least $\sqrt{I}$.
\end{obs}

\begin{proof}
The vertices in $I$ and $\overline{I}$ form a twist and a rainbow\todo{Be more precise}. \ldots
\end{proof}

\begin{obs}\label{cant_separate3}
    If there exists a set $I$ such that $x_I$, $y_I$ and $z_I$ are separated with respect to $<$, then $H_d$ contains a rainbow of size at least $\sqrt[4]{|I|}$.
\end{obs}

\begin{proof}
    By Dilworth's Theorem, if $G$ does not contain a rainbow of size $\sqrt{|I|}$ then there is a subset $I'\subseteq I$ of size at least $\sqrt{|I|}$ such that $x_{I'}$ and $y_{I'}$ form a twist.  Another application of Dilwerth's Theorem then implies that there exits $I''$ of size at least $\sqrt[4]{|I|}$ such that $y_{I''}$ and  $z_{I''}$ form a twist.  But then $x_{I''}$ and $z_{I''}$ form a rainbow.
\end{proof}


\cref{cant_separate} implies that, in any efficient queue layout of $H_d$, the vertex sets of at least two of $G_x$, $G_y$, and $G_z$ must be highly interleaved.  What happens if we just try to interleave $G_x$ and $G_y$ (and leave $G_z$ separated)?  Actually, that also implies a large rainbow.  Use Dilworth to show that (because $H_d$ contains the edge $x_{i}y_i$ is in $H_d$ for each $i\in I$) there exists a subset $I'$ of $I$ so that the ordering of $x_{I'}$ and $y_{I'}$ is the same.  Then apply Dilworth again.

What happens if we try to interleave $G_x$ and $G_z$ but leave $G_y$ separated?  The same problem:  We find $I'$ such that $x_{I'}$ and $z_{I'}$ are reversed.  Therefore one of them forms a big rainbow with $y_{I'}$.  So, ultimately, it must be that $G_x$, $G_y$, and $G_z$ are all highly interleaved.


% \subsection{(Failed) Attempt to Layout $H_d$}
%
% For any $i\in\{1,\ldots,N\}$, let $\overline{\imath}:=N-i+1$. For $v:=x_i$, let $\overline{v}:=z_{\overline\imath}$. For $v:=y_i$, let $\overline{v}:=y_{\overline\imath}$.  We will prove, by induction on $d$ that $H_d$ has an $cd$-queue layout in which $S:=\{v-\overline{v}:v\in V(H_d)\}$  has size $d^{O(1)}$.  The base case $d=1$ is trivial: $H_1$ consists of three copies $H_{x_1},H_{y_1},H_{z_1}$ forming a $3$-cycle. Use the vertex ordering $x_1<y_1<z_1$ and one queue.
%
% For $d>1$, recursively compute a $c(d-1)$-queue layout of $H_{y_1,\ldots,y_N}$.  Use the same layout for $H_{x_1,\ldots,x_N}$ and for $H_{z_1,\ldots,z_N}$ and, in the ordering of $H_V$, use the order $V(H_{x_1,\ldots,x_N}) < V(H_{y_1,\ldots,y_N}) < V(H_{z_1,\ldots,z_N})$.
%
% First we have to show that this layout requires at most $cd$ queues. By induction,  the layouts of the three recursive subgraphs use $c(d-1)$ queues.  The edges that are not contained in one of the three recursive subgraphs can be decomposed into $\lfloor N/6\rfloor$ $6$-cycles of the form $x_i,y_i,z_i,x_{\overline\imath},y_{\overline{\imath}},z_{\overline{\imath}}$ and one $3$-cycle $x_{\ceil{N/2}},y_{\ceil{N/2}},z_{\ceil{N/2}}$.  In the ordering described above, the edges of the $3$-cycle all have length $N$ and four of the edges of each $6$-cycle all have length $n$.  The two other edges of the $6$-cycle have length $2N+y_{i}-y_{\overline{\imath}}$ and $2N-y_{i}+y_{\overline{\imath}}$.\todo{Abusing notation a little here.}
%
%
% STOP: This will fail because we eventually get a recurrence for $|S|$ that looks like $s_d \le 1 + 2s_{d-1}$.  The one counts edges of length $N$. The first $s_{d-1}$ counts the edge lengths that appear in $H_y$.  The second $s_{d-1}$ counts the new edge lengths of the form $2N+y_i-y_{\overline\imath}$.  This resolves to $c^d$ for some $c\ge 2$.
%
% Here's an even easier reason why this can't work: Split the vertices of $G_3$ into two groups: the first half $z_1,\ldots,z_{N/2}$ and the second half  $z_{N/2+1},\ldots,z_N$.  The first half is connected to $x_{N/2}+1,\ldots,x_{N}$ and the second half is connected to $x_{1},\ldots,x_{N/2}$.  All of these edges overlap so they contain a rainbow a twist of size at least $\Omega(\sqrt{n})$.  But a twist in the first set is a rainbow in the second set, so we definitely get a rainbow of size at least $\sqrt{n}$.
%
%
% LESSON 1: To show that $\qn(H_d)\ge 3^{\Omega(d)}$ we could first try to show that the number of edge lengths used in a queue layout of $H_d$ is at least $3^{\Omega(d)}$.    Note that this isn't true for the standard ternary $d$-cube.  There the queue layout only uses edges of length $3^i$ and $2\cdot 3^i$ for $i\in\{0,\ldots,d-1\}$. This turns out not to be so useful (see below).
%
% Lesson 2: The preceding argument shows that we have to interleave $V(G_x)$ and $V(G_z)$ or we have to use different layouts for $G_x$ and $G_z$.  The same goes for $G_y$ and $G_z$.
%
% \begin{compactenum}
%     \item If we use the same (or a mirrored) layout of $G_x$ and $G_z$ and interleave $G_x$ and $G_z$ in the obvious way, so that $x_1<z_1<\cdots<x_N<z_N$ then we immediately get a rainbow.  If we try to reverse $G_z$ so that $x_1<z_N<\cdots<x_N<z_1$ then we're ok, but we end up being constrained to using a \emph{symmetric} layout (where the symmetry is around $y_{N/2}$). This is impossible because interleaving $G_x$ and $G_z$ implies that the layout is not symmetric.
%
%     \item If we try to separate $G_x$ and $G_z$
% \end{compactenum}
%
% % WAIT: We need that the distance $y_i-y_{\overline{\imath}}$ is fixed, independent of $i$ (or comes from a set of size $d^{O(1)}$).  If I really want that difference to be fixed, then there are only $O(N)$ possible orderings. Indeed, once we fix the ordering of $y_1,\ldots,y_{m-1}$ the ordering of $y_{m+1},\ldots,y_N$ is also fixed, and these two sets must perfectly interleave where they overlap.  If we try to put this into the inductive hypothesis, then this breaks everything because $|y_i-y_{\overline\imath}| < N$ but $|x_i-\overline{z_{\imath}}|\ge N$.  Indeed, this completely rules out a recursive layout in which $H_x<H_y<H_z$.
% %
% % Can we make it so that $v-\overline{v}$ falls into a set of size $d^{O(1)}$?  No: If we try to set up a recurrence where $s_d$ tries to count
%
% % we then have to show that $x_i$ and This breaks everything, because it says that $x_i$ and $z_{\not}
% %
% %
% %
% %
% % Now, inductively find a layout of $H_{x_1,\ldots,x_N}$ using $c(d-1)$ queues.  Use the same layout for $H_{y_1,\ldots,y_N}$ and $H_{z_1,\ldots,z_N}$.  Put these three layouts in order.  Now, each of the $6$-cycles described above has $4$ edges of length $N$
%
%
% \begin{figure}
%     \begin{center}
%         \includegraphics{figs/hypertwist_qn-4}
%     \end{center}
%     \caption{The ordering used in an efficient queue layout of $H_3$.}
%     \label{hypertwist_qn}
% \end{figure}
%
%
% \subsection{A New Edge-Length-Based Parameter}
%
% Let $M$ be a $2k$-vertex perfect matching with edge set $\{(a_i,b_i):i\in\{1,\ldots,k\}\}$ and let $<$ be a total order of $V(M)$ such that $a_i < b_j$ for all $i,j\in\{1,\ldots,k\}$.  The \emph{length} of an edge $vw$ in $M$ is $\ell_<(vw):=1+|\{x\in V(M): v < x< w\}|$.  Define the \emph{length set} $L(M):=\{\ell(vw):vw\in E(M)$.
%
% Clearly, if $M$ contains a rainbow (with respect to $<$) of size $r$, then $|L(M)|\ge r$.  I believe there is also an inequality in the other direction:
%
% \begin{conj}
%     There exists $a,c >0$ such that, if $|L(M)|\ge r$ then $M$ contains a rainbow of size at least $ar^{c}$.
% \end{conj}
%
% The preceding conjecture is false.  Let $k$ be an even integer and use the order $a_1<\cdots<a_k$ and $b_1<b_{k/2+1}<b_2<b_{k/2+2}<\cdots< b_{k-1}<b_k$.  Then the largest rainbow has size $2$ since $a_ib_i$ and $a_jb_j$ don't nest unless $i \le k/2$ and $j\ge k/2$.  Furthermore, for $i\le k/2$, $\ell(i):=\ell(a_ib_i)=k-i+(2i-1)=k+i-1$ is an increasing function of $i$, so $|L(M)|\ge k/2$.
%
% Actually for $i > k/2$, $\ell(i):=\ell(a_ib_i)=k-i+2(i-k/2)=i$. This function takes on all values in the set $\{f(k/2+1)\ldots,f(k)\}=\{k/2+1,\ldots,k\}$. So, actually, $|L(M)|=k-1$.  (If we use an odd $k$, then we can get $|L(M)|=k$.)
%
% This gives an alternative way to do bipartite matchings that has lots of different edge lengths and small rainbows.  Maybe this can be used to find a queue layout of $H_d$?
%

\section{A shortcut}

Let $X:=\{x_0,\ldots,x_{n-1}\}$, $Y:=\{y_0,\ldots,y_{n-1}\}$ and $Z:=\{z_0,\ldots,z_{n-1}\}$.  Let $<$ be a total order on $X\cup Y\cup Z$ where $X<Y<Z$ and, for each $0\le i<j< n$, $x_i<x_j$, $y_i<y_j$ and $z_i<z_j$.
Take a tripartite graph $G:=(X,Y,Z,E)$ such that $G$ has an $s$-stack layout $(<,\varphi)$ and each of $G_{XY}:=G[X\cup Y]$, $G_{YZ}:=G[Y\cup Z]$ and $G_{XZ}:=G[X\cup Z]$ are bipartite 2-sided $\delta$-expanders.

A result of Bourgain and Yehudahoff shows that such a $G$ exists width $s\in O(1)$.  That result is hard to understand. An easier construction, due to Alon and Roichman shows that if we just pick $k:=\lceil c(\delta)\log n\rceil$ random values $\ell_1,\ldots,\ell_k\in\{0,\ldots,n-1\}$, then the graph $G_{XY}$ with edge set $E(G_{XY})=\{(x_i,y_{(i+\ell_j)\bmod n}):i\in\{0,\ldots,n-1\},j\in\{1,\ldots,k\}\}$ is a $\delta$-expander.  It's easy to see that this ``shift-graph'' has a stack layout $(<,\varphi)$ using $s:=2k= O(\log n)$ stacks.  This will be good enough if we can prove that the queue number of $G$ is at least $n^\alpha$ for some fixed $\alpha>0$.  (The remark on Page~8 of the Alon-Roichman paper even considers the cyclic group $Z_n$, which is what we're using here.)

The idea now is to show that any $q$-queue layout $(<',\varphi')$ of $G_{XY}$ is quite restricted.  Basically, that $X$ can be separated into $t:=O(q)$ groups $X_1,\ldots,X_t$ such that $X_1<'\cdots<'X_t$ and $<'$ essentially agrees with $<$ or $<'$ essentially reverses $<$ within each group $X_i$ (with the same decision for each $X_i$).  This would be good enough, since we can then argue as follows:
\begin{enumerate}
   \item Assume, without loss of generality that $<'$ essentially agrees with $<$.
   \item To avoid a large rainbow in $G_{XY}$, $<'$ must essentially reverse $<$ on $Y$.
   \item To avoid a large rainbow in $G_{XY}$, $<'$ must essentially agree with $<$ on $Z$.
   \item But now $<'$ essentially agrees with $<$ or $X$ and $Z$, which forces a large rainbow in $G_{XZ}$.
\end{enumerate}
\bibliographystyle{plainurlnat}
\bibliography{qn-vs-sn}


\section{Starting Fresh}

Let $A:=\{a_1,\ldots,a_n\}$ and $B:=\{b_1,\ldots,b\}$ be two disjoint sets and let $<$ be a total order over $A\cup B$ such that $a_1<\cdots<a_n<b_1\cdots<b_n$.  A \emph{$2$-sided $d$-monotone $(A,B)$-$(1+\epsilon)$-expander} is a bipartite graph $G$ with vertex set $V(G):=A\cup B$ such that
\begin{compactenum}[(P1)]
    \item $G$ has no $(d+1)$-rainbow with respect to $<$; and
    \item For any $S\subset A$, $|N_G(S)\cap B|\ge\min\{n,(1+\epsilon)|S|\}$.
\end{compactenum}

\begin{thm}[Bourgain]\label{bourgain}
    For any $\epsilon>0$, there exists an integer $d:=d(\epsilon)$ such that for any $n\in\N$ there exists a $2$-sided $d$-monotone $(A,B)$-$(1+\epsilon)$-expander.
\end{thm}


\begin{lem}
    Let $G$ be a $2$-sided $d$-monotone $(A,B)$-$c$-expander given by \cref{bourgain} and let $<$ be a $q$-queue ordering of $G$. Then there exists a partition $\mathcal{P}$ of $\{1,\ldots,n\}$ such that
    \begin{compactenum}[(i)]
        \item for each $P\in\mathcal{P}$ and each $i,j\in P$, $i<j$ if and only if $a_i<a_j$; or
        \item for each $P\in\mathcal{P}$ and each $i,j\in P$, $i<j$ if and only if $a_i>a_j$.
    \end{compactenum}
\end{lem}


\end{document}
