\documentclass{patmorin}
\listfiles
\usepackage{pat}
\usepackage{paralist}
\usepackage{dsfont}  % for \mathds{A}
\usepackage[utf8x]{inputenc}
\usepackage{skull}
\usepackage{paralist}
\usepackage{graphicx}
\usepackage[noend]{algorithmic}

\usepackage[normalem]{ulem}
\usepackage{cancel}
\usepackage{enumitem}

\usepackage{todonotes}

\usepackage[longnamesfirst,numbers,sort&compress]{natbib}

\usepackage[mathlines]{lineno}
\setlength{\linenumbersep}{2em}
% \linenumbers
% \rightlinenumbers
% \linenumbers
\newcommand*\patchAmsMathEnvironmentForLineno[1]{%
 \expandafter\let\csname old#1\expandafter\endcsname\csname #1\endcsname
 \expandafter\let\csname oldend#1\expandafter\endcsname\csname end#1\endcsname
 \renewenvironment{#1}%
    {\linenomath\csname old#1\endcsname}%
    {\csname oldend#1\endcsname\endlinenomath}}%
\newcommand*\patchBothAmsMathEnvironmentsForLineno[1]{%
 \patchAmsMathEnvironmentForLineno{#1}%
 \patchAmsMathEnvironmentForLineno{#1*}}%
\AtBeginDocument{%
\patchBothAmsMathEnvironmentsForLineno{equation}%
\patchBothAmsMathEnvironmentsForLineno{align}%
\patchBothAmsMathEnvironmentsForLineno{flalign}%
\patchBothAmsMathEnvironmentsForLineno{alignat}%
\patchBothAmsMathEnvironmentsForLineno{gather}%
\patchBothAmsMathEnvironmentsForLineno{multline}%
}


\newcommand{\coloured}[2]{{\color{#1}{#2}}}
\newenvironment{colourblock}[1]{\color{#1}}{}

\newcommand{\condref}[1]{(C\ref{#1})}

% Taken from
% https://tex.stackexchange.com/questions/42726/align-but-show-one-equation-number-at-the-end
\newcommand\numberthis{\addtocounter{equation}{1}\tag{\theequation}}


\setlength{\parskip}{1ex}


\DeclareMathOperator{\diam}{diam}
\DeclareMathOperator{\tw}{tw}
\DeclareMathOperator{\gm}{gm}
\DeclareMathOperator{\gs}{gs}
\DeclareMathOperator{\stw}{stw}
\DeclareMathOperator{\ltw}{ltw}
\DeclareMathOperator{\pw}{pw}
\DeclareMathOperator{\lpw}{lpw}
\DeclareMathOperator{\lhptw}{lhp-tw}
\DeclareMathOperator{\lhppw}{lhp-pw}

\DeclareMathOperator{\x}{x}
\DeclareMathOperator{\depth}{d}
\DeclareMathOperator{\sh}{cbt}
\DeclareMathOperator{\cbt}{cbt}
\DeclareMathOperator{\sgn}{sgn}
\DeclareMathOperator{\dc}{dc}
\DeclareMathOperator{\afci}{\overline{\chi}_\pi}
\DeclareMathOperator{\afcn}{\dot{\chi}_\pi}

\newcommand{\ellt}{{\lfloor\ell/2\rfloor}}

\title{\MakeUppercase{Queue Number versus Stack Number}\thanks{This research was partly funded by NSERC.}}
\author{Pat Morin%
    \thanks{School of Computer Science, Carleton University}\quad
    and Friends}

\date{}

\DeclareMathOperator{\ddiv}{div}
\DeclareMathOperator{\hist}{h}

\newcommand{\colored}[2]{{\color{#1}#2}}

\usepackage{tabularx}

\DeclareMathOperator{\ci}{\overline{\pi}}

\begin{document}

% \begin{titlepage}
\maketitle

\begin{abstract}
    These are some notes I'm keeping on trying to show that queue number is (or isn't) bounded by stack number.
\end{abstract}
% \end{titlepage}

% \pagenumbering{roman}
% \tableofcontents
%
% \newpage
% \pagenumbering{arabic}



\section{Introduction}

An old result of \citet[Theorem 8]{dujmovic.wood:stacks} shows that queue number is (polynomially) bounded by by stack number if and only if bipartite 3-stack graphs have bounded queue number.  Even forgetting the bipartite condition, this means we can restrict ourselves to graphs made up of three outerplanar graphs $G_1$, $G_2$, and $G_3$ having a common outer face $v_1,\ldots,v_n$.

If we're trying to show that queue number is not bounded by stack number then we will need all three of these triangulations.  Why? Because two triangulations of $v_1,\ldots,v_n$ form a (Hamiltonian) planar graph.  We already know that any planar graph has queue number at most $49$.

I conjecture that the expected queue number of the random (multi)graph $G$ obtained by taking each of $G_1,\ldots,G_3$ to be a random triangulation of $v_1,\ldots,v_n$ is unbounded.  Here, a \emph{random triangulation} of a $n$-vertex cycle $C$ is the outerplanar graph with outer face $C$ and whose weak dual is a uniformly random binary search tree with $n-2$ nodes.  [There is a bijection between these triangulations and full binary trees with $n-1$ leaves, so the number of such triangulations is the $(n-2)$-th Catalan number $C_{n-2}$.]

Even if this turns out to be false or if the result can be proven with a deterministic construction, I think this model of random graph is interesting and worth studying.  In general, we could define the ``random stack-number-$k$ graph'' $R_{n,k}$ obtained by taking the union of $k$ random triangulations of the cycle $v_1,\ldots,v_n$ and ask questions about it.
\begin{compactenum}
    \item What is the expected maximum degree of $R_{n,k}$?
    \item What is the expected diameter of $R_{n,k}$?
    \item What is the expected treewidth of $R_{n,k}$?
    \item What is the expected queue number of $R_{n,k}$?
\end{compactenum}

Some conjectures about the answers to these questions:

\begin{enumerate}
    \item For $k=1$, the answers are $\Theta(\log n)$, $\Theta(\sqrt{n})$, $2$, and $O(1)$.  (The last two are facts, not conjectures)

    \item For $k=2$, the answers are $\Theta(\log n)$, $\Theta(\sqrt{n})$, $\Theta(\sqrt{n})$, and $O(1)$, respectively.  (Only the last one is a fact, the rest are conjectures.)

    \item For $k=3$ the answers are $\Theta(\log n)$, $\Theta(n^{1/3})$, $\Theta(n^{2/3})$ and $\Omega(1)$. (All conjectures.)
\end{enumerate}

The question about maximum-degree is really just about uniformly random binary trees.  It asks what is the longest path in such a tree that repeatedly goes from a node to its right child.  It feels like we can use symmetry to treat this the same as a random walk in an arbitrary binary tree.  In an arbitrary binary tree, the probability that a random walk takes more than $\log n+k$ steps is at most $2^{-k}$.  I guess this could easily be made precise and settles the first question for any constant $k$.

Here's a very rough intuition about the diameter (for $k=1$).  Since the height of the random tree $T$ that defines the triangulation $R_1$ is $\Theta(\sqrt{n})$, most nodes of $T$ have a child that is the root of a subtree of size $O(\sqrt{n})$.  So most of the time, the triangle $v_1v_iv_n$ of $R_1$ has $\max\{i, n-i\}\in O(\sqrt{n})$, say $i\in O(\sqrt{n})$.  Then the node furtest from $v_1$ is at distance least one greater than the node furthest from $\{v_i,v_n\}$ on the subgraph induced by $v_i,\ldots,v_n$.  This is a subgraph of size $n-O(\sqrt{n})$.  I guess this argument is basically showing that there exists a cutset of size $2$ that separates a vertex from a part of the graph of size $n-\sqrt{n}$.

Here's another approach for the diameter in the case $k=1$.  With high probability, the tree $T$ contains a root-to-leaf path $P$ of length $\Omega(\sqrt{n})$.  With high probablity, $P$ contains $\Omega(\sqrt{n})$ \emph{alternations}; subpaths of the form $xyz$ where $y$ is the right child of $x$ and $z$ is the left child of $y$ (or \emph{vice-versa} with the roles of left and right reversed).  This means there exists a vertex $v$ of $R_1$ such that $\{v_1,v_n\}$ is separated from $v$ by a sequence $\{a_1,b_1\},\ldots,\{a_k,b_k\}$ of disjoint $2$-vertex cutsets, where $k\in\Omega(\sqrt{n})$.  So every path from $v_1$ to $v$ must contain at least one of $a_i$ or $b_i$ for each $i\in\{1,\ldots,k\}$.  Therefore, $\diam(G)\ge k+1\in \Omega(\sqrt{n})$.

Both of the preceding arguments seems to break down immediately, even for $k=2$.  Note that understanding diameter is a prerequisite for understanding treewidth since (at least for $k=2$) $\tw(R_2) \in O(\diam(R_2))$ since $R_2$ is planar.
% In fact, upon reflection, I think that $\diam(R_2)\in O(\log n)$.  












\bibliographystyle{plainurlnat}
\bibliography{qn-vs-sn}


\end{document}
